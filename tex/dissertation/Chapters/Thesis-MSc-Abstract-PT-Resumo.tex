% #############################################################################
% RESUMO em Português
% !TEX root = ../main.tex
% #############################################################################
% use \noindent in firts paragraph
% reset acronyms
\acresetall
\noindent A segmentação por expressão de referência é uma tarefa fundamental em visão computacional que integra a compreensão de linguagem natural com localização visual precisa. A aplicação desta tarefa a imagens aéreas apresenta desafios únicos devido às altas densidades de objectos e complexidades geográficas. Introduzimos o Aerial-D, o maior dataset de segmentação por expressão de referência para imagens aéreas até à data, compreendendo 37.288 patches de imagem com 1.522.523 expressões de referência cobrindo 259.709 alvos anotados através de objectos individuais, grupos e categorias semânticas abrangendo 21 classes distintas desde veículos e infraestruturas até tipos de cobertura do solo. O dataset representa a primeira pipeline de construção totalmente automática neste campo, utilizando geração sistemática de expressões baseada em regras seguida de melhoramento por Large Language Model que enriqueceu significativamente tanto a variedade linguística como a riqueza de detalhes visuais das expressões de referência. Treinámos um modelo no Aerial-D juntamente com benchmarks aéreos anteriores, produzindo segmentação unificada de instâncias, semântica e histórica a partir de texto, desempenhando fortemente em todos os datasets que avalhámos. Disponibilizamos publicamente o Aerial-D e os modelos em \href{https://huggingface.co/datasets/luisml77/aerial-d}{\texttt{huggingface.co/datasets/luisml77/aerial-d}} e a pipeline completa e código em \href{https://github.com/luispl77/aerialseg}{\texttt{github.com/luispl77/aerialseg}}.