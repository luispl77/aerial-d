%%%%%%%%%%%%%%%%%%%%%%%%%%%%%%%%%%%%%%%%%%%%%%%%%%%%%%%%%%%%%%%%%%%%%%
%     File: ExtendedAbstract_abstr.tex                               %
%     Tex Master: ExtendedAbstract.tex                               %
%%%%%%%%%%%%%%%%%%%%%%%%%%%%%%%%%%%%%%%%%%%%%%%%%%%%%%%%%%%%%%%%%%%%%%

\begin{abstract}

Referring expression segmentation is a fundamental task in computer vision that integrates natural language understanding with precise visual localization of target regions. Applying this to aerial imagery presents unique challenges because spatial resolution varies widely across datasets, targets often shrink to only a few pixels, and scenes contain very high object densities. This work presents Aerial-D, a large-scale referring expression segmentation dataset for aerial imagery comprising 37,288 image patches with 1,522,523 referring expressions covering 259,709 annotated targets across individual objects, groups, and semantic categories spanning 21 distinct classes from vehicles and infrastructure to land-cover types. The dataset is constructed through a fully automatic pipeline that combines systematic rule-based expression generation with Large Language Model enhancement, enriching both the linguistic variety and visual detail within the referring expressions. As an additional capability, the pipeline produces dedicated historic counterparts for each scene, supporting real-world archival analyses such as monitoring urban change across decades. A single model is trained on Aerial-D together with prior aerial datasets, yielding unified instance, semantic, and historic segmentation from text, with the historic branch demonstrating robustness to monochrome, sepia, and grainy degradations that appear in archival aerial photography. The dataset, trained models, and complete pipeline are publicly available at \href{https://huggingface.co/datasets/luisml77/aerial-d}{\texttt{huggingface.co/datasets/luisml77/aerial-d}} and \href{https://github.com/luispl77/aerialseg}{\texttt{github.com/luispl77/aerialseg}}.

\vspace{0.3cm}


\end{abstract}
