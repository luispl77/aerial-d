%%%%%%%%%%%%%%%%%%%%%%%%%%%%%%%%%%%%%%%%%%%%%%%%%%%%%%%%%%%%%%%%%%%%%%
%     File: ExtendedAbstract_intro.tex                               %
%     Tex Master: ExtendedAbstract.tex                               %
%%%%%%%%%%%%%%%%%%%%%%%%%%%%%%%%%%%%%%%%%%%%%%%%%%%%%%%%%%%%%%%%%%%%%%

\section{Introduction}
\label{sec:intro}

Referring Instance Segmentation is a fundamental task in computer vision that requires models to identify and segment specific object instances using natural language descriptions. When applied to aerial photographs, also referred to as Referring Remote Sensing Instance Segmentation (RRSIS), this task represents a major challenge due to the intricate characteristics of aerial imagery, including varying scales and resolutions of top-down perspectives, geographic complexities, extreme object density variations, and unique spatial relationships not present in ground-level photography.

\begin{figure}[H]
\centering
\includegraphics[width=\columnwidth]{./images/6samples.png}
\caption{Representative examples from Aerial-D dataset showing diverse referring expressions with corresponding aerial images and ground truth masks.}
\label{fig:dataset_examples}
\end{figure}

A critical component for developing effective models for RRSIS is access to high-quality datasets containing aerial photographs, precise segmentation masks, and natural referring expressions. To address this need, we introduce Aerial-D, the largest referring segmentation dataset for aerial imagery to date, comprising over 1.5 million expressions across 37,288 aerial image patches. The primary focus of Aerial-D is not only to provide exceptional training data quality but also to establish the most challenging benchmark for RRSIS to date, which will hopefully prompt research into novel RRSIS model architectures that can surpass current performance limitations.

Our key contributions include: (1) We present Aerial-D, the first comprehensive large-scale aerial referring expression segmentation dataset. (2) We introduce a novel fully automatic pipeline that transforms simple instance segmentation datasets into rich referring segmentation resources by leveraging large language models, demonstrating the effectiveness of model distillation and QLoRA techniques for enabling large-scale dataset generation with substantial cost savings. (3) We train and evaluate a baseline model on Aerial-D and conduct cross-dataset evaluation to establish comprehensive benchmark scores for future research comparison. (4) We demonstrate how historic image filters can augment training data and present a final comprehensive model trained on Aerial-D plus four additional existing datasets with historic filtering applied, showing capabilities for referring instance segmentation, referring group segmentation, class-based segmentation, land cover segmentation, and robust handling of historic imagery including black and white, grainy, or sepia photographs.

