%%%%%%%%%%%%%%%%%%%%%%%%%%%%%%%%%%%%%%%%%%%%%%%%%%%%%%%%%%%%%%%%%%%%%%
%     File: ExtendedAbstract_intro.tex                               %
%     Tex Master: ExtendedAbstract.tex                               %
%%%%%%%%%%%%%%%%%%%%%%%%%%%%%%%%%%%%%%%%%%%%%%%%%%%%%%%%%%%%%%%%%%%%%%

\section{Introduction}
\label{sec:intro}

Referring expression segmentation is a computer vision task in which a model receives a natural language description of a target region and must return the corresponding segmentation mask. Because the phrasing can reference any concept, the task is open-vocabulary and the target can be a single instance, a coherent group of instances, or an entire semantic category, such as "all roads in the patch" or "the vegetation strip along the river". The remote-sensing literature coined the term Referring Remote Sensing Instance Segmentation (RRSIS)~\cite{yuan2023rrsis} for the subset that limits expressions to single instances, while later datasets like NWPU-Refer~\cite{yang2024large} incorporated group-level expressions and Aerial-D extends the coverage further to include instances, groups, and full land-cover classes. When this formulation is applied to aerial photographs, although we refer to it simply as referring expression segmentation throughout this article, the problem becomes especially demanding because top-down perspectives compress object scales, spatial resolution varies across sensors, many targets occupy only a handful of pixels, and the scenes themselves contain extreme object densities. In many real deployments, analysts revisit archival aerial surveys to study how cities or coastlines evolved. To support that use case, the pipeline also models the monochrome, sepia, and grainy degradations found in historic imagery so the resulting models can handle tasks such as assessing long-term urban change.

A critical component for developing effective models for RRSIS is access to high-quality datasets containing aerial photographs, precise segmentation masks, and natural referring expressions. To address this need, this work presents Aerial-D, a large-scale referring expression segmentation dataset for aerial imagery comprising 1,522,523 expressions across 37,288 aerial image patches, compared with prior RRSIS datasets~\cite{yuan2023rrsis,liu2024rotated,yang2024large}. Figure~\ref{fig:dataset_examples} highlights how this corpus spans rural and urban scenes and objects, land-cover regions, groups of multiple objects, and entire categories while retaining unrestricted, richly worded referring expressions tailored to each target. Each sample further includes an automatically generated historic counterpart, enabling models trained on the dataset to understand black-and-white, grainy, or sepia aerial imagery alongside contemporary captures.

% Locally tighten spacing for this figure to avoid a visible gap
\begingroup
\setlength{\intextsep}{6pt}
\setlength{\abovecaptionskip}{2pt}
\setlength{\belowcaptionskip}{0pt}
\begin{figure}[H]
\centering
\includegraphics[width=\columnwidth]{./images/6samples.png}
\caption{Representative examples from Aerial-D dataset showing diverse referring expressions with corresponding aerial images and ground truth masks.}
\label{fig:dataset_examples}
\end{figure}
\endgroup


The key contributions of this work include: (1) a comprehensive toolchain that enables the production of complex referring expression datasets from instance segmentation datasets, including a rule-based pipeline, Large Language Model enhancement and distillation methods, and historic image data augmentation with dedicated filtering; (2) the construction of Aerial-D, a dataset comprising over 1.5 million expressions across 37,288 aerial image patches, created entirely through the proposed automatic pipeline; and (3) a unified model trained on Aerial-D alongside four additional datasets, applying the full toolchain—including historic transformations across all training data—to deliver referring expression segmentation over instances, groups, classes, and land cover regions while maintaining reliable performance on degraded historic imagery typical of archival aerial surveys.
