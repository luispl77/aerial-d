% #############################################################################
% This is Chapter 5
% !TEX root = ../main.tex
% #############################################################################
% Change the Name of the Chapter i the following line
\fancychapter{Evaluation Setup}
\cleardoublepage
% The following line allows to ref this chapter
\label{chap:evaluation}

This chapter presents the experimental setup for evaluating open-vocabulary aerial image segmentation models. We describe the model architecture, training procedures, evaluation metrics, and baseline comparisons used in our study.

% #############################################################################
\section{Model Architecture}

Our approach is based on the RSRefSeg (Referring Segmentation with Rule-based annotation System) architecture, which combines text and visual understanding for precise object segmentation in aerial imagery.

\begin{figure}[H]
\centering
\includegraphics[width=0.9\textwidth]{./Images/RSRefSeg.png}
\caption{RSRefSeg Architecture Overview}
\label{fig:rsrefseg_architecture}
\end{figure}

% #############################################################################
\section{ClipSAM Model Implementation}

We implement a ClipSAM model that leverages SigLIP for multimodal understanding and SAM for precise segmentation mask generation.

\begin{figure}[H]
\centering
\includegraphics[width=\textwidth]{./Images/clipsam.png}
\caption{ClipSAM architecture overview showing the integration of SigLIP2 vision-language encoder with SAM mask decoder through custom prompter networks for text-guided segmentation. The dual-pathway design processes both local (token-level) and global (sentence-level) text-visual interactions to generate sparse and dense prompts for precise aerial image segmentation.}
\label{fig:clipsam_architecture}
\end{figure}

\subsection{Dataset Statistics}

% Dataset examples figure
\begin{figure}[H]
\centering
\includegraphics[width=\textwidth]{./Images/dataset.png}
\caption{Representative examples from AERIAL-D dataset showing diverse referring expressions with corresponding aerial images and ground truth masks.}
\label{fig:dataset_examples}
\end{figure}

% Dataset statistics table
\begin{table}[H]
\centering
\caption{Dataset Statistics Summary}
\label{tab:dataset_stats}
\begin{tabular}{@{}lrrr@{}}
\toprule
\textbf{Metric} & \textbf{Train} & \textbf{Val} & \textbf{Total} \\
\midrule
Total Patches & 32,460 & 11,054 & 43,514 \\
Individual Objects with Expressions & 94,179 & 34,536 & 128,715 \\
Individual Expressions & 651,098 & 244,210 & 895,308 \\
Groups with Expressions & 99,986 & 34,216 & 134,202 \\
Group Expressions & 487,214 & 163,472 & 650,686 \\
Total Samples & 1,138,312 & 407,682 & 1,545,994 \\
Avg. Expressions per Individual Object & 6.91 & 7.07 & 6.96 \\
Avg. Expressions per Group & 4.87 & 4.78 & 4.85 \\
\bottomrule
\end{tabular}
\end{table}

\subsection{Category Distribution}

% Category distribution table
\begin{table}[H]
\centering
\caption{Object Category Distribution by Instance Type and Source Dataset}
\label{tab:category_dist}
\resizebox{\textwidth}{!}{%
\begin{tabular}{@{}lrrrrr@{}}
\toprule
\textbf{Category} & \textbf{Individual Instances} & \textbf{Groups} & \textbf{Instance Expressions} & \textbf{Group Expressions} & \textbf{Source Dataset} \\
\midrule
Ship & -- & -- & -- & -- & iSAID \\
Large Vehicle & -- & -- & -- & -- & iSAID \\
Small Vehicle & -- & -- & -- & -- & iSAID \\
Building & -- & -- & -- & -- & iSAID \\
Storage Tank & -- & -- & -- & -- & iSAID \\
Harbor & -- & -- & -- & -- & iSAID \\
Swimming Pool & -- & -- & -- & -- & iSAID \\
Tennis Court & -- & -- & -- & -- & iSAID \\
Soccer Ball Field & -- & -- & -- & -- & iSAID \\
Roundabout & -- & -- & -- & -- & iSAID \\
Basketball Court & -- & -- & -- & -- & iSAID \\
Bridge & -- & -- & -- & -- & iSAID \\
Ground Track Field & -- & -- & -- & -- & iSAID \\
Plane & -- & -- & -- & -- & iSAID \\
Helicopter & -- & -- & -- & -- & iSAID \\
Building & -- & -- & -- & -- & LoveDA \\
Water & -- & -- & -- & -- & LoveDA \\
Barren Land & -- & -- & -- & -- & LoveDA \\
Agricultural Area & -- & -- & -- & -- & LoveDA \\
Forest Area & -- & -- & -- & -- & LoveDA \\
Road & -- & -- & -- & -- & DeepGlobe \\
\bottomrule
\end{tabular}%
}
\end{table}

% #############################################################################
\section{Experimental Configuration} 
This section describes the training configuration and experimental setup for evaluating the RSRefSeg model on the AerialD dataset.

\begin{figure}[h]
\centering
\includegraphics[width=0.8\textwidth]{./Images/test_env}
\caption{Test Environment}
\label{fig:test_env}
\end{figure}

\subsection{Expression Type Analysis}

Table~\ref{tab:expression_type_dist} shows the distribution of different expression types generated across the pipeline, including rule-based expressions and LLM enhancements.

% Expression type distribution table
\begin{table}[H]
\centering
\caption{Expression Type Distribution}
\label{tab:expression_type_dist}
\resizebox{\textwidth}{!}{%
\begin{tabular}{@{}ccccccr@{}}
\toprule
\textbf{Category} & \textbf{Position} & \textbf{Extreme} & \textbf{Size} & \textbf{Color} & \textbf{Relationship} & \textbf{Total Count} \\
\midrule
\multicolumn{7}{l}{\textbf{Rule-Based Individual Instance Expressions}} \\
\midrule
\checkmark & & & & & & -- \\
\checkmark & \checkmark & & & & & -- \\
\checkmark & \checkmark & & & & \checkmark & -- \\
\checkmark & & \checkmark & & & & -- \\
\checkmark & \checkmark & \checkmark & & & & -- \\
\checkmark & \checkmark & \checkmark & & & \checkmark & -- \\
\checkmark & \checkmark & & \checkmark & & & -- \\
\checkmark & \checkmark & & \checkmark & & \checkmark & -- \\
\checkmark & \checkmark & \checkmark & \checkmark & & & -- \\
\checkmark & \checkmark & \checkmark & \checkmark & & \checkmark & -- \\
\checkmark & & & & \checkmark & & -- \\
\checkmark & \checkmark & & & \checkmark & & -- \\
\checkmark & \checkmark & & & \checkmark & \checkmark & -- \\
\checkmark & & \checkmark & & \checkmark & & -- \\
\checkmark & \checkmark & \checkmark & & \checkmark & & -- \\
\checkmark & \checkmark & \checkmark & & \checkmark & \checkmark & -- \\
\checkmark & \checkmark & & \checkmark & \checkmark & & -- \\
\checkmark & \checkmark & & \checkmark & \checkmark & \checkmark & -- \\
\checkmark & \checkmark & \checkmark & \checkmark & \checkmark & & -- \\
\checkmark & \checkmark & \checkmark & \checkmark & \checkmark & \checkmark & -- \\
\midrule
\multicolumn{7}{l}{\textbf{Rule-Based Group Expressions}} \\
\midrule
\checkmark & \checkmark & & & & & -- \\
\checkmark & \checkmark & \checkmark & & & & -- \\
\checkmark & \checkmark & & & & \checkmark & -- \\
\checkmark & \checkmark & & & & \checkmark & -- \\
\checkmark & & & & & & -- \\
\checkmark & & & \checkmark & & & -- \\
\bottomrule
\end{tabular}%
}
\end{table}

\subsection{LLM Enhancement Statistics}

% LLM enhancement stats table
\begin{table}[H]
\centering
\caption{LLM Enhancement Expression Distribution}
\label{tab:llm_enhancement_stats}
\begin{tabular}{@{}lrrr@{}}
\toprule
\textbf{Expression Source} & \textbf{Train} & \textbf{Val} & \textbf{Total} \\
\midrule
Rule-Based Expressions & -- & -- & -- \\
LLM Enhanced (Language Variations) & -- & -- & -- \\
LLM Unique (Visual Details) & -- & -- & -- \\
\midrule
\textbf{Total Expressions} & \textbf{--} & \textbf{--} & \textbf{--} \\
\bottomrule
\end{tabular}
\end{table}

The experimental setup evaluates model performance across different expression types and enhancement levels to understand the impact of each component on segmentation accuracy.

\begin{table}[htb]
\centering
\normalsize
    \caption{Network Link Conditioner Profiles}
    \label{tab:network_profiles}
{\footnotesize
    \begin{tabular}{ | c | c | c | c | }
    \hline 
    \textbf{Network Profile}	& \textbf{Bandwidth} & \textbf{Packets Droped} & \textbf{Delay}\\ \hline \hline
    Wifi  & 40 mbps  &  0\%  &   1 ms \\ \hline
    3G  & 780 kbps  &  0\%  &   100 ms \\ \hline 
    Edge  & 240 kbps  &  0\%  &   400 ms \\ \hline
    \end{tabular}
    }
\end{table}

\subsection{Training Configuration}

% [Training parameters, optimization settings, data splits, augmentation strategies, etc.]

\subsection{Evaluation Methodology}

% [Metrics, validation splits, cross-dataset testing, ablation study design, etc.]
% #############################################################################
\section{Proin ornare dignissim lacus}
Pellentesque habitant morbi tristique senectus et netus et malesuada fames ac turpis egestas. Vestibulum tortor quam, feugiat vitae, ultricies eget, tempor sit amet, ante. Donec eu libero sit amet quam egestas semper. Aenean ultricies mi vitae est. Mauris placerat eleifend leo. Quisque sit amet est et sapien ullamcorper pharetra. Vestibulum erat wisi, condimentum sed, commodo vitae, ornare sit amet, wisi. Aenean fermentum, elit eget tincidunt condimentum, eros ipsum rutrum orci, sagittis tempus lacus enim ac dui. Donec non enim in turpis pulvinar facilisis. Ut felis.

Et ``optimistic'' nulla dui purus, eleifend vel, consequat non, dictum porta, nulla. Duis ante mi, laoreet ut, commodo eleifend, cursus nec, lorem. Aenean eu est. Etiam imperdiet turpis. Praesent nec augue. Curabitur ligula quam, rutrum id, tempor sed, consequat ac, dui $G_j$, nec ligula et lorem consequat ullamcorper $p$ ut mauris eu mi mollis luctus $j$, porttitor ut, \Cref{unchoke_gain}, uctus posuere justo:

\begin{description}
  \item[$N_j$] Is the number of times peer $j$ has been optimistically unchoked.
  \item[$n_j$] Among the $N_j$ unchokes, the number of times that peer $j$ responded with unchoke or supplied segments to peer $p$.
  \item[$C_{r[j]}$] The cooperation ratio of peer $j$. If peer $j$ never supplied peer $p$, the information of $C_{r[j]}$ may not be available.
  \item[$C_{r (max)}$] The maximum cooperation ratio of peer $p$’s neighbors, i.e., $C_{r (max)} = max(C_r)$.
\end{description}

\begin{equation}
\label{unchoke_gain}
 G_j =
  \begin{dcases}
    \frac{n_j C_{r[j]}}{N_j} &\quad \text{if } n_j > 0\\
    \frac{C_{r (max)}}{N_j + 1} &\quad \text{if } n_j = 0
  \end{dcases}
\end{equation}

Cursus $C_{r (max)}$ conubia nostra, per inceptos hymenaeos $j$ gadipiscing mollis massa $N_j = 0$, unc ut dui eget nulla venenatis aliquet $G_j = C_{r (max)}$.

Vestibulum accumsan eros nec magna. Vestibulum vitae dui. Vestibulum nec ligula et lorem consequat ullamcorper. Class aptent taciti sociosqu ad litora torquent per conubia nostra, per inceptos hymenaeos. Phasellus eget nisl ut elit porta ullamcorper. Maecenas tincidunt velit quis orci. Sed in dui. Nullam ut mauris eu mi mollis luctus. Class aptent taciti sociosqu ad litora torquent per conubia nostra, per inceptos hymenaeos. Sed cursus cursus velit. Sed a massa. 

Both \Cref{fig:tx_layer_4,fig:tx_layer_5} Phasellus eget nisl ut elit porta ``perfect'' tincidunt. Class aptent taciti sociosqu ad litora torquent per conubia nostra.

\begin{figure}[h]
%\centering
       \subfigure[Adaptation System Test 4]{\label{fig:tx_layer_4}\includegraphics[width=0.5\textwidth]{./Images/tx_layer_4}}  
   %    \centering 
       \subfigure[Adaptation System Test 5]{\label{fig:tx_layer_5}\includegraphics[width=0.5\textwidth]{./Images/tx_layer_5}}   
        \caption{Adaptation System Behavior Test}
        \label{fig:fig:adapt_behave_2}
\end{figure}

Cras sed ante. Phasellus in massa. Curabitur dolor eros, gravida et, hendrerit ac, cursus non, massa. Aliquam lorem. In hac habitasse platea dictumst. Cras eu mauris. Quisque lacus. Donec ipsum. Nullam vitae sem at nunc pharetra ultricies. Vivamus elit eros, ullamcorper a, adipiscing sit amet, porttitor ut, nibh. Maecenas adipiscing mollis massa. Nunc ut dui eget nulla venenatis aliquet. Sed luctus posuere justo. Cras vehicula varius turpis. Vivamus eros metus, tristique sit amet, molestie dignissim, malesuada et, urna.