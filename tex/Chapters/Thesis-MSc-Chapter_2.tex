% #############################################################################
% This is Chapter 2
% !TEX root = ../main.tex
% #############################################################################
% Change the Name of the Chapter i the following line
\fancychapter{Fundamental Concepts}
\cleardoublepage
% The following line allows to ref this chapter
\label{chap:back}

This chapter introduces the fundamental technical concepts and architectures underlying open-vocabulary referring segmentation systems.

% #############################################################################
\subsection{Deep Learning Fundamentals}

Basic concepts of deep neural networks, convolutional architectures, and optimization techniques.

% #############################################################################
\subsection{Transformer Architecture}

Self-attention mechanisms, encoder-decoder architectures, and positional encoding.

% #############################################################################
\subsection{Vision-Language Models}

Multimodal architectures that process both visual and textual information.

\subsubsection{CLIP Architecture}

Contrastive Language-Image Pre-training model architecture and training methodology.

\subsubsection{SigLIP Improvements}

Enhancements over CLIP including sigmoid loss and improved training efficiency.

% #############################################################################
\subsection{Segmentation Models}

Deep learning approaches for image segmentation tasks.

\subsubsection{Segment Anything Model (SAM)}

Promptable segmentation architecture and zero-shot capabilities.

% #############################################################################
\subsection{Referring Expression Segmentation}

Task definition, challenges, and architectural approaches for language-guided segmentation.