% #############################################################################
% This is Chapter 3
% !TEX root = ../main.tex
% #############################################################################
% Change the Name of the Chapter i the following line
\fancychapter{Related Work}
\cleardoublepage
% The following line allows to ref this chapter
\label{chap:architecture}

This chapter reviews existing work in aerial image segmentation datasets, referring segmentation models, and related approaches.

% #############################################################################
\subsection{Aerial Image Segmentation Datasets}

Overview of datasets used for aerial imagery analysis tasks.

\subsubsection{Semantic Segmentation Datasets}

Datasets providing pixel-level land cover and land use classifications.

\subsubsection{Instance Segmentation Datasets}

Datasets with individual object boundaries and identities for aerial objects.

\subsubsection{Referring Instance Segmentation Datasets}

Existing datasets that combine aerial imagery with natural language referring expressions.

% #############################################################################
\subsection{Referring Segmentation Models}

Previous approaches to language-guided segmentation in natural and aerial images.

% #############################################################################
\subsection{Aerial Imagery Analysis Methods}

Traditional and deep learning approaches for aerial image understanding.

% #############################################################################
\subsection{Large Language Models for Vision}

Applications of LLMs to vision tasks and multimodal understanding.